\documentclass[
 reprint, %preprint for single col, double spacing
 %pre, % others journal styles: pra, prl, etc
 amsmath, amssymb, aps, % preload maths packages
 a4paper,
 %onecolumn
]{revtex4-2}

\usepackage{microtype}
\usepackage{bm}
\usepackage{lipsum} % just to fill text
\usepackage{dcolumn} % for decimal alignment in tables
\usepackage{siunitx}
\usepackage{physics}

\usepackage{graphicx}
\usepackage{subfigure}

\usepackage{natbib}
% the style you should use will depend on what is expected. 
% Imagine they want:
% numbered, square brackets for the citations
% name of authors, article title, journal name, and DOI if available
\bibliographystyle{unsrtnat}
%\bibliographystyle{unsrt}
%\bibliographystyle{siam}
%\bibliographystyle{apsrev4-2}

% revtex already loads hyperref but explicit import is needed to make cleveref work
\usepackage[hidelinks]{hyperref}
\usepackage{cleveref}

\begin{document}

\title{RevTex example}

\author{Jeremy Worsfold}
\email{j.worsfold@unsw.edu.au}
\affiliation{
    School of Physics,
    University of New South Wales - Sydney 2052, Australia
}

\date{\today}

\begin{abstract}
\lipsum[1]

\end{abstract}


\maketitle


\section{Introduction}
\label{sec:intro}

\lipsum

\section{Tables and figures}

You can force tables and figures to be in specific areas using the \texttt{[h]} option but generally, I think it's best to let the compiler find where it should go.
Other options include \texttt{[t]} for top of the page, or \texttt{[b]} for bottom of the page. Usually in journals like APS, figures go at the top.
You can also just move it about in the text to move the figure/table onto another page. But make sure it's close or directly after being referred to in the text.

\begin{table}
    \centering
    \begin{tabular}{cccc}
         Col 1 (\unit{\newton\meter})& Col 2  & Col 3  & Col 4 \\
         \colrule
         1 & 2 & 3 & 4 \\
         5 & 6 & 7 & 8 \\
    \end{tabular}
    \caption{Caption}
    \label{tab:my_label}
\end{table}

\begin{table}
\begin{ruledtabular}
\begin{tabular}{lcdr}
    Left & Centered & \multicolumn{1}{c}{Decimal} & Right\\
    \colrule
    1 & 2 & 3.001 & 4 \\
    10 & 20 & 30 & 40 \\
    100 & 200 & 300.0 & 400 \\
\end{tabular}
\end{ruledtabular}
\caption{An example table}
\label{tab:table1}
\end{table}


\section{Other Packages}


\subsection{SI units}

An example use of units $\SI{52.3}{\newton}=\SI{52.3}{\kg\per\meter\squared}$.

\subsection{physics package}

\begin{equation}
    \int_{-\infty}^\infty e^{-x^2} d x = \sqrt{\pi} \label{eq:gaussInt}
\end{equation}

\begin{equation}
    \int_{-\infty}^\infty e^{-x^2} \dd x = \sqrt{\pi}
\end{equation}

\begin{equation}
    \frac{\partial^2u}{\partial x^2} = \pdv[2]{u}{x} \label{eq:partialDiffExample}
\end{equation}

\section{referencing text}
\label{sec:ref}

This is a reference to Section \ref{sec:intro}. This is a neater reference to \cref{sec:intro} or \Cref{sec:intro}.
It also works for equations such as \cref{eq:gaussInt}, as opposed to Eq. \eqref{eq:gaussInt}.
However, if you directly mention that the thing you're referring to is an equation, ODE, PDE etc, then it's probably better practice to just give the equation number. 
For example, substituting \cref{eq:gaussInt} into the PDE \eqref{eq:gaussInt} and simplifying, we obtain...

\subsection{Citations}

I recommend you use \texttt{natbib} to do referencing, this works well with the \texttt{revtex} documentclass which also comes with a bibliography style. 
In this `vancouver' referencing style, you only refer using a number so the only command you really need is \texttt{cite}. For instance, this is a nod to a great mathematician \cite{turing1990chemical}. 
Citing multiple sources at the same time just requires separating each key with commas \cite{turing1990chemical,cooley1967fft}.

Give your entries in the bibliography useful keys and try to keep to a convention, e.g. \texttt{<lead author><date><first word of title>}.
Reference managers (and google scholar) will automatically do things like this. Wrapping text with \{ \} brackets ensures words stay capitalized.


\begin{figure*}
    \centering
    \subfigure{\includegraphics[width=8.6cm]{figures/fig_1.eps}}
    \hfill
    \subfigure{\includegraphics[width=8.6cm]{figures/fig_1.eps}}
    \caption{An example of sub-figures}
    \label{fig:PDMP}
\end{figure*}



% should have the bibliography before the appendices generally.
\bibliography{ref}

\appendix

\section{First appendix}
The numbering of equations and sections changes after the appendix
\begin{equation}
    e^{i\pi}\equiv-1.
\end{equation}

\end{document}